\documentclass[a4paper,11pt]{article}
\usepackage[margin=3cm]{geometry}

\title{Report for Final Project IP Autumn 2026}
\date{Deadline: 29th Jan.  23:59}

\author{Alex, Komari, Natalia, and Penelope}

\begin{document}

\maketitle

\section*{Task Description}

The project involves developing an interpreter to manage a small  programming language that can be used by a controller to operate washing machines. The language allows the user to manipulate a simple integer array stored in limited memory space. The programming language includes operations (memory allocation and deallocation, arithmetic/logical operations, and printing) which need to be recognized and executed, while returning runtime errors as specified. The program needs to be subdivided into different modules built on top of each other; each module should be able to treat the functions of the module below as black box. This allows for independent development and testing of each module. 

\section*{Structure of the Project}
Each module was created based on our task breakdown:
\begin{itemize}
    \item main.c: Read the input file given by user and process line by line. Return runtime errors.
    \item interpreter.c: Parse and interpret each line (3-character command, parameter 1 and parameter 2) and select operations.
    \item memory.c: responsible for managing the memory required by the mini-language. It provides functions to allocate and free contiguous memory blocks as well as to read and modify certain cells in memory. 
    \item variables.c: Store and keep track of variables and their memory locations.
    \item functions.c: Program operations required for the programming language.

\end{itemize}

To make use of the modularity of Abstract Data Types (ADTs), each module was accompanied by a header file that reveals only the necessary functionalities.

\section*{Data Structures and Implementation Choices}

Explain and justify the implementation choices that you made and the main data structures used in the project.
\begin{itemize}
    \item Memory: the data structures used are a fixed-size array of 100 integer cells and a linked list to track free memory. Each node of the lost stores the start position and length of a contiguous free block in memory. This implementation was chosen because it allows efficient (de)allocation without scanning the full memory array, while sorted insertion enables easy combination of adjacent free blocks and limits fragmentation.
    \item Variables: ///
\end{itemize}

\section*{Testing}

Explain how you tested your code.Here you should explain and justify the testing strategies that you used. Do not just say "I used the tester program". 

\section*{Self-Reflection}

Explain what you learned during this project



\end{document}
